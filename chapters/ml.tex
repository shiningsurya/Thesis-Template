\chapter{Machine Learnings}
\label{ch:ml}

\par This chapter describes the Machine Learning (ML) / Artificial Intelligence (AI) system in the \vf which identifies triggers worthy of a closer look. First and foremost, the motivation is discussed in~\autoref{sec:coh}.
The ML model is introduced and described in~\autoref{sec:ml}. 
Lastly, the newly \emph{learnt} ML model is applied to the whole of \vfpfs and results are examined in~\autoref{sec:ml_vfps}.

\section {Coherent analysis}
\label{sec:coh}

\par \vf operates on an \emph{incoherent} level. In all of the pipeline, only the power is considered.
There is no phase considered anywhere. This lack of phase greatly simplifies the pipeline design but comes at a cost.
Hence the name, \emph{incoherent}.

\par For an array of $N$ antennas, co-addition in \emph{incoherent} (see~\autoref{ssub:needcoaddition}) only boosts the signal (or specifically, the signal-to-noise, \sn) by $\sqrt{N}$. 
An \emph{coherent} analysis would boost the \sn~by $N$.

\par In addition to the \sn~boost, a \emph{coherent} analysis would also provide good localization capability.
This localization ability is of great value since \frb~s are known to be originating from outside the galaxy by which small angular variations could lead to astronomically large separated distances due to large radii.

\subsection{Coherent analysis with \vf}

\par In order to be able to do \emph{coherent} analysis, voltage data has to be recorded. 
A second of raw voltage data consisting of two polarization measures about $250$ MB. 
A typical \vf compute node has $450$ GB of dedicated Solid State Disk (SSD). Meaning, a \vf compute can only have $1800$ s (or $30$ minutes) of raw voltage data before getting filled.
Given the extremely large volume of voltage data, it is impractical to record voltages all the time.
Hence, \vf performs searches \emph{incoherently} and triggers voltages for a coherent follow-up analysis.

\par It is also not practical to trigger voltages on all the triggers. It would be so as if voltages are recorded all the time since the trigger rates are high.
So only a subset of the triggers received are to be allowed to trigger voltages. 
Naturally, such a subset has to be selected on the basis of the signal's merit of being a real signal of astrophysical origin.
The question of how to decide what triggers to trigger on is the main goal of this chapter.

\section {Machine Learning}
\label{sec:ml}

\par The objective of the AI is to given a trigger, identify if a following \emph{coherent} analysis has to be followed.
In an ideal situation, it would be desirous to have an AI solution which can identify all the true signals (i.e., the signals of interest).
However, training such an AI solution is extremely difficult.
The signals of interest are rare. Hence, any dataset produced would posses this assymmetric, which then would make training difficult.
Hence, this approach is abandoned. 
Succiently, the mission of the AI is to select those triggers which are worthy of a second closer look.

\par The major hurdle here is the lack of a labeled dataset. 
Surely, \vfpfs possesses a large amount of data but not even a fraction of it is labeled.
Meaning signals are identified to be true signals of interest or RFI or whitenoise.
Labeling the whole dataset or a small subset is a laborious task.
Instead, an alternative strategy is employed.

Firstly, a convolutional autoencoder (CAE,~\cite{cae}) is trained on the entire dataset. 
And, with a small subset of triggers from known pulsars and known RFI, a classifier feeding on the encoding layer is trained.
This way, bulk of the training is done in an un-supervised fashion and the classifier is trained on a small subset.

\subsection{Model}
\label{ssub:mlmode}
\par For the convolutional autoencoder (CAE, henceforth), the entire dataset is used.


\subsection{Dataset}
\label{ssub:dataset}


\par All the known pulsars triggers from the \vfpfs are added to the true class. 
The high activity RFI triggers are added to the false class.
In addition, the whitenoise



\section {Applying ML to VFPS}
\label{sec:ml_vfps}

\par Having the newly \emph{learnt} set of AI models, a retrospect analysis of all the triggers part of \vfpfs is performed to uncover any previously missed triggers.
Results are 
