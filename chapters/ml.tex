\chapter{Machine Learnings}
\label{ch:ml}

\par This chapter describes the Machine Learning (ML) / Artificial Intelligence (AI) system in the \vf which identifies triggers worthy of a closer look. First and foremost, the motivation is discussed in~\autoref{sec:coh}.
The ML model is introduced and described in~\autoref{sec:ml}. 
Lastly, the newly \emph{learnt} ML model is applied to the whole of \vfpfs and results are examined in~\autoref{sec:ml_vfps}.

\section {Coherent analysis}
\label{sec:coh}

\par \vf operates on an \emph{incoherent} level. In all of the pipeline, only the power is considered.
There is no phase considered anywhere.
Hence the name, \emph{incoherent}.
This lack of phase greatly simplifies the pipeline design but comes at a cost.

\par For an array of $N$ antennas, co-addition in \emph{powers (incoherently)} (see~\autoref{ssub:needcoaddition}) only boosts the signal (or specifically, the signal-to-noise, \sn) by $\sqrt{N}$. 
A \emph{coherent} analysis would boost the \sn~by $N$. The reason being the phase information.
This can also be appreciated with this toy example. Consider the following equations:
\begin{align*}
{\rm sin} (\theta) + {\rm cos} (\phi) = {\rm sin} (\theta) + {\rm sin} (\frac{pi}{2} - \phi)  &= 2 {\rm sin} (\frac{\pi}{4} + \frac{\theta - \phi}{2}) {\rm cos} (-\frac{\pi}{4} + \frac{\theta + \phi}{2}) \\
%{\rm sin} (\theta) + {\rm cos} (\phi) &= \sqrt{2} {\rm sin} (\frac{\pi}{4} + \theta) \\
|| {\rm sin}(\theta) || ^2  + || {\rm cos}(\phi) || ^2 &= \big( {\rm sin} (\theta) + {\rm cos(\phi)} \big)^2 - 2{\rm sin}(\theta){\rm cos}(\phi)
\end{align*}
If the phase difference was zero, the coherent equation would yield \sqrt{2} gain whereas power equation would only yield trivial gain.
In this way, coherent mode of analysis gives more sensitivity but the trade off is that there should be no phase difference. 
This makes coherent analysis extremely challenging.

\par In addition to the \sn~boost, a \emph{coherent} analysis would also provide good localization capability.
The localization ability is a by-product of relative phase estimation.
Given a set of relative phases, one can translate to geometric delays and from there geometric path differences from asrophysical sources.
This ability is of great value since \frb~s are known to be originating from outside the galaxy by which small angular variations could lead to astronomically large separated distances due to large radii.

\subsection{Coherent analysis with \vf}

\par In order to be able to do \emph{coherent} analysis, voltage data has to be recorded. 
A second of raw voltage data consisting of two polarization measures about $250$ MB. 
A typical \vf compute node has $450$ GB of dedicated Solid State Disk (SSD). Meaning, a \vf compute can only have $1800$ s (or $30$ minutes) of raw voltage data before getting filled.
Given the extremely large volume of voltage data, it is impractical to record voltages all the time.
Hence, \vf performs searches \emph{incoherently} and triggers voltages for a coherent follow-up analysis.

\par It is also not practical to trigger voltages on all the triggers. It would be so as if voltages are recorded all the time since the trigger rates are high.
So only a subset of the triggers received are to be allowed to trigger voltages. 
Naturally, such a subset has to be selected on the basis of the signal's merit of being a real signal of astrophysical origin.
The question of how to decide what triggers to trigger on is the main goal of this chapter.

\section {Machine Learning}
\label{sec:ml}

\par The objective of the AI is to given a trigger, identify if a following \emph{coherent} analysis has to be initiated.
In an ideal situation, it would be desirous to have an AI solution which can identify all the true signals (i.e., the signals of interest).
However, training such an AI solution is extremely difficult.
The signals of interest are rare. Hence, any dataset produced would posses this assymmetric, which then would make training difficult.
Hence, this approach is abandoned. 
Succiently, the mission of the AI is to select those triggers which are worthy of a second closer look.

\par The major hurdle here is the lack of a labeled dataset. 
Surely, \vfpfs possesses a large amount of data but not even a fraction of it is labeled.
Labeled meaning signals identified to be true signals of interest or RFI or whitenoise.
Labeling the whole dataset is an impossible task.
Instead, an alternative strategy is employed which is discussed in~\autoref{ssub:strategy}.
The datasets are described in~\autoref{ssub:dataset}.

\subsection{Strategy}
\label{ssub:strategy}

\par To make full use of the data available, a convolutional autoencoder (\autoref{sssub:cae}) is first trained on the entire dataset.
And, with a small subset of triggers from known pulsars and known RFI, a classifier feeding on the encoding layer is trained.
This way, bulk of the training is done in an un-supervised fashion and the classifier is trained on a small subset.

\subsubsection{Convolutional AutoEncoder (CAE)}
\label{sssub:cae}
\par TODO

\subsection{Dataset}
\label{ssub:dataset}

\par The entire \vfps dataset is used for training the convolutional autoencoder. 
Triggers from known radio pulsars constitute a part of the \texttt{TRUE} class.
All the injected and received triggers also constitute the \texttt{TRUE} class.
RFI is manually selected using the creiterion described in~\autoref{sec:rfi}.
The \dm{150} RFI is also selected and both form the \texttt{FALSE} class.
The actual breakdown of the cardinalities are given in~\autoref{tab:dataset}.

% Please add the following required packages to your document preamble:
% \usepackage{booktabs}
\begin{table}[]
\label{tab:dataset}
	\begin{tabular}{@{}lll@{}}
		\toprule
		Dataset & Class & Number \\ \midrule
		VFPS & - & 818 848 \\ \midrule
		Injected & True & 8 480 \\
		Pulsars & True & 23 574 \\
		TRUE & - & 32 054 \\ \midrule
		RFI & False & 8 207 \\
		DM150 RFI & False & 20 660 \\
		FALSE & - & 28 867 \\ \bottomrule
	\end{tabular}
	\caption{The breakdown of \vfps dataset used for ML. See text.}
\end{table}

\section {Applying ML to VFPS}
\label{sec:ml_vfps}

\par TODO
